\documentclass[twocolumn]{article}

\usepackage{amstext}
\usepackage{amsmath}
\usepackage{graphicx}
\usepackage{float}
\usepackage{caption}
\usepackage{subcaption}
\usepackage[margin=1in, paperwidth=8.5in, paperheight=11in]{geometry}
%\usepackage{gensymb}
\usepackage{here}
\usepackage{caption}

\usepackage[backend=biber]{biblatex}
\addbibresource{references.bib}

\def\changemargin#1#2{\list{}{\rightmargin#1\leftmargin#2}\item[]}
\let\endchangemargin=\endlist
\begin{document}

\title{Wolbachia, Toxoplasma and O. Cordyceps: Extreme Host-Manipulating Microbes That Are Changing the World}
    \author{C. Bekins, R. Chapman and B. Wasti}
    \date{December 2015}
    \maketitle

\section*{Introduction}
When thinking of infection most people will think of getting sick or of a swelling cut that is turning green. However, most people do not think about wandering around randomly and trying to climb a tree, only to have a fungus grow out of their spinal column. That is because humans are (mostly) lucky with respect to infections in this world; the worst we have to worry about is dying.

For other animals, dying is possibly the best case for an infection. There are a few microbes that will manipulate their host to extreme lengths. From hijacking the reproductive systems to complete mind control, and seemingly everything in between, the parasitic microbial world has developed a whole range of tools for which to survive at any cost. In this paper, we will go over three of these host-manipulating microbes, each with a different characteristic: \textit{Wolbachia}, the reproductive-system hijacker; \textit{Toxoplasma}, the subtle schizophrenic inducer; and the infamous \textit{O. Cordyceps}, the mind-controlling fungus. 

\section*{Wolbachia}
\textit{Wolbachia} is a group of bacteria that infects arthropod species. Technically there is only one true type of \textit{Wolbachia}, \textit{Wolbachia pipientis}, but due to the genetic diversity of different strains, it is commonly split into two groups, group A and group B, although it is generally just referred to as \textit{Wolbachia}.

The bacteria is found in the ovaries and testes of a wide range of arthropods.\cite{Wbio} It is passed on via reproduction, and can only be spread through females. \textit{Wolbachia} are considered mutualistic endosymbionts, and are "required for survival of their hosts."\cite{Wdisc_nem} An interesting thing to note is how universal \textit{Wolbachia}seems to be. It has been demonstrated that \textit{Wolbachia} infects 25-70\% of species of insects.\cite{Wdisc_nem}

To give a bit of background,\textit{Wolbachia} was first discovered by Samuel Burt Wolbach and described by Marshall Hertig in 1924.\cite{Winit} Wolbach had been studying the cause of Rocky Mountain Spotted Fever and by 1916 had shown that \textit{Dermacentor andersoni}, a tick native to the Rocky Mountain area, was the transmitter of the disease.\cite{wolbachia} He was unable to grow the actual bacteria causing the disease, a species of \textit{Rickettsia}, in a cell-free culture leading "him to speculate as to the relationship between the cells of the host and the intracellular parasites."\cite{wolbachia} This is important to note, because the discovery of \textit{Wolbachia} is closely linked to this idea.

\begin{figure}[!ht]
    \centering
    \includegraphics[width=.4\textwidth]{images/Wolbachia.png}
    \caption{\textit{Wolbachia} bacteria (Green circles) inside a host cell. It is difficult to tell that there is a bacteria present if one is unaware.\cite{Wwiki_image} }
    \label{fig:wolbachiha_tree}
\end{figure}

Hertig came to work with Wolbach, and together they investigated a bacteria living in the gonads of the \textit{C. pipiens} mosquito. This bacteria is known today as \textit{Wolbachia pipientis} which Hertig gave a detailed description of in 1936.\cite{Wdiscription} At this point scientists did not know of the gene manipulation that \textit{Wolbachia} has on it's hosts, and the discovery largely went unnoticed.

During the 1950s a group of scientists found that certain matching between \textit{Culex} mosquitos produced no offspring. They named this phenomenon cytoplasmic incompatibility, although they were not sure as to what the cause was.\cite{Wcyto_iso} It wasn't until the 1970s that cytoplasmic incompatibility was linked to \textit{Wolbachia}, when a group of scientists eliminated the \textit{Wolbachia} through antibiotics.\cite{Wcyto_cause}

Also during the 1960s and '70s scientists noted "unusual structures in the oocytes or in the hypodermis" of filarial nematodes but did not associate them with bacteria.\cite{wolbachia} It wasn't until 1995 that these "unusual structures" were determined to be \textit{Wolbachia}.\cite{Wstruct} 

The history of \textit{Wolbachia} is rather curious, and for good reason. On the surface, \textit{Wolbachia} are just harmless bacteria living in the reproductive areas of different insects that don't really do anything. But only recently has it been shown that \textit{Wolbachia} has shaped the world in drastic ways, and is one of the most prolific parasitic bacteria.\cite{Wdistr}

There is a very good reason for the abundance of \textit{Wolbachia}, for better or for worse, it completely hijacks the host organism's reproductive organs. One of the most unique features of \textit{Wolbachia} is the reproductive changes it makes in its host species. It can cause cytoplasmic incompatibility \cite{Wci0}\cite{Wci1}\cite{Wci2}\cite{Wci3}, parthenogenesis \cite{Wparth}, feminization of males \cite{Wfem} and male killing.\cite{Wmale_killing} All of these different changes force the host species to pass on \textit{Wolbachia} to their offspring, also known as vertical transfer.

%##### How does it function?
The most prominent of these reproductive changes is cytoplasmic incompatibility which is a "reproductive incompatibility between sperm and egg."\cite{Wbio} \textit{Wolbachia} can only be passed on through eggs, and not sperm, so in order to get around this limitation it has evolved to not allow inter-breeding between members of the same species where only one of them is infected with \textit{Wolbachia}. However, it gets even more complicated than this.

There are actually two types of cytoplasmic incompatibility, unidirectional and bidirectional. Unidirectional allows an infected female to mate with an uninfected male, which makes sense because the \textit{Wolbachia} can still be passed on. Bidirectional cytoplasmic incompatibility occurs when there is no case where an infected member and an uninfected member can mate succesfuly. The reason for these two different 'modes' is even more complicated still.

It goes back to the idea that there are two separate groups of \textit{Wolbachia}, group A and group B. Comparing the 16S rDNA does not give a full picture for the genetic diversity of the \textit{Wolbachia}, and the ftsZ gene was used to compare the genetics of 38 different \textit{Wolbachia} strains.\cite{Wgenetics} It was found that there is in fact a very large genetic difference between strains, and that group A and group B diverged 58-67 million years ago. Figure \ref{fig:wolbachia_tree} shows a part of the tree for the strain distribution of Wolbachia.

\begin{figure}[!ht]
    \centering
    \includegraphics[width=.4\textwidth]{images/WolbachiaTree.jpg}
    \caption{ A depiction of the strain distribution for Wolbachia. Most importantly, the separation between group A and group B is depicted. \cite{Wtree_image} }
    \label{fig:wolbachia_tree}
\end{figure}

Going back to cytoplasmic incompatibility, when a species is infected with two different strains of \textit{Wolbachia} (a species can in fact be infected with multiple strains) that are mutually incompatible, bidirectional cytoplasmic incompatibility occurs.\cite{WbiCI} It makes more sense when discussing how \textit{Wolbachia} cause this cytoplasmic incompatibility.

The exact mechanisms for the cause of cytoplasmic incompatibility are not known. What is known is that the \textit{Wolbachia} in the male modify the sperm in such a way that only if the same strain of \textit{Wolbachia} is in the female species to "recover" the sperm will correct fertilization occur.\cite{Wbio}

In many ways \textit{Wolbachia} acts like an encryption algorithm, "encrypting" the sperm sent to the female, and only if the female has the correct decryption algorithm can she get pregnant. This is also why different strains of \textit{Wolbachia} are incompatible, using a wrong decryption algorithm will generate nonsense.

The other unique reproductive change \textit{Wolbachia} causes is parthogenisis, or the ability for unfertilized eggs to grow into healthy adults essentially eliminating the need for males in a species. This makes sense, \textit{Wolbachia} can only be passed on through females, so it does not care about the males. But then one might realize that \textit{Wolbachia} took over the male's job, and took millions of years of evolution for sexual differentiation and threw it out the window.

Perhaps the most famous case of parthogenisis caused by \textit{Wolbachia} is in the \textit{Trichogramma} wasps, where all members of the species are female. Weirdly, by treating the wasps with antibiotics to kill of the \textit{Wolbachia} cause "some female parthenogenetic strains ... to revert to production of male progeny."\cite{Wpar_removal} It would appear that \textit{Wolbachia} is merely surpressing the need for male's in certain species. Even stranger is the fact that "phylogenetic evidence suggests that [parthogenisis] has evolved several times independently in these bacteria" which, as one scientist puts it, suggests "a simple biochemical mechanism."\cite{Wbio}  

Despite the fact that we still do not know the exact biochemical mechanism that causes parthogenisis, we do know that it is the result of gamete duplication.\cite{Wgamete_duplication} There is still work to be done in order to figure out how \textit{Wolbachia} is inducing gamete duplication.

In fact, there is a lot of research that needs to be done on \textit{Wolbachia} in general. There is still so much that is not known about the bacteria, even how many organisms it infects is a mystery. However, there is another microbe which might be even more mysterious than the \textit{Wolbachia}.

\section*{Toxoplasma}

\section*{Cordyceps}
Cordyceps, or more specifically \textit{Ophiocordyceps unilateralis}, is a fungus that infects ants. Otherwise known as a specialized fungal parasite. This zombifying, mind-controlling microbe is able to modify the behavior of its host. Eventually it kills its host in a location optimal for its reproduction. 

The story of how \textit{Ophiocordyceps} functions is a cycle, so you could start talking about it anywhere, but from an ant's perspective everything starts when it comes into contact with the spores of the fungus. The spores recognize they are on a host, and form a biological drill that utilizes enzymes and mechanical pressure to breach the ant's tough exoskeleton. Once inside, the fungus transforms into a yeast-like state, living in the hemocoel of the ant.\cite{cordy_infection} This is where the highly specialized adaptations of \textit{Ophiocordyceps} start to shine. There are many things that are unknown about how \textit{Ophiocordyceps} manipulates the mind of its host, but the outcome has been studied in depth. \textit{Ophiocordyceps} leads its host ant to the northern side of a sapling approximately 25 cm above the ground. There, the ant closes it mandibles on the bark of the sapling or the main vein on the bottom side of a leaf, never to open them again. It is there that the fungus rapidly colonizes the host, restructuring all nutrients available inside the host to produce a large fruiting body which grows out of the back of the ants head that will produce the spores to restart the cycle.\cite{life_of_dead_ant}

\begin{figure}[!ht]
    \centering
    \includegraphics[width=.4\textwidth]{images/cordyceps_ant.jpg}
    \caption{The spore bearing fruiting body of \textit{Ophiocordyceps unilateralis} growing out of the back of the head of an ant whose mandibles are tightly gripping a twig.\cite{cordy_video} }
    \label{fig:cordyceps_ant}
\end{figure}

\textit{Ophiocordyceps unilateralis} was first discovered in 1859 by the British naturalist Alfred Russel Wallace who, probably not coincidently, is best known for independently conceiving the theory of evolution through natural selection. Darwin published his now famous "On the Origin of Species" in the same year, which gives some good context as to where the scientific paradigms were at the time.\cite{darwin} 


\section*{Impacts And Analysis}

Scientists have only recently begun doing research on the three microbes discussed above. However, the three microbes mentioned have had a massive impact on the world as we know it. 


A cousin of the ant manipulating fungus, \textit{Cordyceps sinensis} has been a cornerstone of Chinese medicine for centuries. It has been claimed it can be used to treat many things including respiration and pulmonary diseases, renal, liver, and cardiovascular diseases, hypo sexuality, and hyperlipidemia. Most of these purported benefits have yet to be sufficiently investigated since interest from the western world only began to increase in the last two decades. \cite{medicinal_cordy} Another Cordyceps species, \textit{Cordyceps subsessilus} was used to create the immunosuppresive drug cyclosporin, which is widely used in organ transplantation. The drug prevents the rejection of the foreign organ by interfering with the growth of T cells and the activity of the immune system.\cite{cordy_tcells}

The parasitic Cordyceps fungi are key to the diversity of the ecosystems they live in. If any one population of insects begins to become dominant in the ecosystem, it becomes an easy target for a Cordyceps fungus to infect.  \cite{cordy_video}

- Toxoplasma has been correlated to a rise in schizophernia

The presence of \textit{Wolbachia} in filarial nematodes has raised interesting questions about how \textit{Wolbachia} evolves in its host species. Some scientists discovered that \textit{Wolbachia} "have possibly been lost during evolution along some lineages of filarial nematodes."\cite{Wevolution_loss} The reason this discovery is bizare is because \textit{Wolbachia}'s ability to evolve into the life-cycles of so many different species makes sense if \textit{Wolbachia} gives the infected a fitness advantage over uninfected members. However, if species evolve to get rid of \textit{Wolbachia} then this points to the fact that \textit{Wolbachia} does not supply them with a fitness advantage.

So why would one species derive an advantage from a \textit{Wolbachia} infection whereas another species might not? The answer is that nobody really knows, at least not yet. The impact this discovery has had on our understanding of evolution has not truly been felt yet. However, when the reason for differences in derived fitness advantage from \textit{Wolbachia} infections is understood it could uncover a greater understanding of how interactions between organic matter works. It might explain why some people like coffee and find it helps them get work done, while others do not even like the smell of coffee.

Aside from helping us understand the world better, \textit{Wolbachia} may have the ability to help cure humans of deadly diseases. A recent study found that \textit{Aedes aegypti} mosquitos infected with \textit{Wolbachia} cause it to no longer be able to carry the pathogen that causes Dengue Fever.\cite{Wdengue_fever}

This is big news, because it might lead to the eradication of Dengue Fever, for which there is currently no vaccine. By artificially bringing \textit{Wolbachia} into the \textit{Aedes aegypti} species the fear of Dengue Fever could vanish entirely. Not only does this impact those in Dengue Fever risk areas, but it may also lead to the prevention of other diseases as well. Many scientists are looking into the idea of infecting mosquitos that harbor \textit{Milaria} with \textit{Wolbachia} as a prevention technique.\cite{Wmilaria}

Using \textit{Wolbachia} as an "antidote" for different diseases is novel and has potential to change the world dramatically. The best part is that it is a one-time use antidote, because \textit{Wolbachia} has the tools to infect an entire species through vertical transmission. It may also cause scientists to look at other symbiotic relationships between organisms, and see if they have any potential to help humans. There might be a new era of using living organisms as tools to cure diseases and help humans in other ways.

The abuse of the relationship between \textit{Wolbachia} and its host may also be done in the opposite way. A species infected with \textit{Wolbachia} becomes dependant on \textit{Wolbachia}. If the \textit{Wolbachia} were to be killed off, say by antibiotics, then the host would struggle to survive. This idea can be used to treat filarial infections in humans: targetting the \textit{Wolbachia} kills off the filaraea.\cite{wolbachia}\cite{Wcure_filarial_infection}

Although this technique is rather new and is not widely used yet, it has potential to be a whole new route by which we can treat filarial infections. By combining antiwolbachia medication with antifilarial treatments, there is twice the potential to successfully cure the infection. Through studying \textit{Wolbachia}, we can save lives.


\section*{The Efficacy of Adding These Microbes to the Six Microbes Class Syllabus}

- Assess scientific aspects

- Assess potential as a laboratory study

- Historical impacts over time

- Overall significance and impact

- Etc.


\printbibliography

\end{document}
